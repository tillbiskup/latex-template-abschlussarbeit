% Inoffizielle LaTeX-Vorlage für Bachelor-/Masterarbeiten
% an der Fakultät für Chemie und Pharmazie
% der Albert-Ludwigs-Universität Freiburg
%
% Diese Vorlage ist lediglich ein Vorschlag, der versucht, sowohl
% typografischen Ansprüchen ansatzweise gerecht zu werden als auch
% nutzbar zu sein.
%
% Jegliche Nutzung auf eigene Verantwortung.
%
% Copyright (c) 2018, Till Biskup

%% Mathematischer Formelsatz
\usepackage{amsmath}
\ifPDFTeX
\else
  \usepackage{unicode-math}
\fi  


% Vektoren: fett, kursiv
\ifPDFTeX
  \usepackage{bm}
  \renewcommand*{\vec}[1]{\bm{#1}}
\else
  \AtBeginDocument{\renewcommand*{\vec}[1]{\symbfit{#1}}}
\fi

% Tensoren: serifenlos, fett, kursiv
\ifPDFTeX
  \DeclareMathAlphabet{\mathbfsf}{\encodingdefault}{\sfdefault}{bx}{sl}
  \newcommand*{\tens}[1]{\mathbfsf{#1}}
\else
  \newcommand*{\tens}[1]{\symbfsfit{#1}}
\fi	

% Differentialoperator: aufrecht
\ifPDFTeX
  \newcommand*{\upd}{\mathrm{d}}
\else
  \newcommand*{\upd}{\symup{d}}
\fi

% mathematische Konstanten: aufrecht
\ifPDFTeX
  \newcommand*{\upe}{\mathrm{e}}
  \newcommand*{\upi}{\mathrm{i}}
  \usepackage[Symbolsmallscale]{upgreek}
\else
  \newcommand*{\upe}{\symup{e}}
  \newcommand*{\upi}{\symup{i}}
  \newcommand{\uppi}{\symup{\pi}}
\fi

% Operatoren mit Dach
\newcommand*{\op}[1]{\hat{#1}}

% Dirac-Notation
\newcommand*{\bra}[1]{\left\langle#1\right\rvert}
\newcommand*{\ket}[1]{\left\lvert#1\right\rangle}
\newcommand*{\braket}[2]{\langle#1\lvert#2\rangle}
