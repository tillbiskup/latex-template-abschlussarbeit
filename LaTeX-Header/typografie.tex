% Inoffizielle LaTeX-Vorlage für Bachelor-/Masterarbeiten
% an der Fakultät für Chemie und Pharmazie
% der Albert-Ludwigs-Universität Freiburg
%
% Diese Vorlage ist lediglich ein Vorschlag, der versucht, sowohl
% typografischen Ansprüchen ansatzweise gerecht zu werden als auch
% nutzbar zu sein.
%
% Jegliche Nutzung auf eigene Verantwortung.
%
% Copyright (c) 2018, Till Biskup

%% Mikrotypografie
\usepackage{microtype}

%% Anführungszeichen sprachabhängig und "intelligent"
\usepackage[autostyle]{csquotes}

%% Abbildungsüber- und Tabellenunterschriften
\addtokomafont{caption}{\small}       % kleinere Schrift
\addtokomafont{captionlabel}{\textbf} % fette Schrift
\setcapindent{0ex}                    % Keine hängenden Über-/Unterschriften

%% typografisch korrekte Tabellen
\usepackage{booktabs}
\KOMAoptions{captions=tableheading} % Tabellen haben Überschriften

%% Zeilenabstand erhöhen
\usepackage{setspace}
\setstretch{1.1}               % geringfügig erhöhter Zeilenabstand
%\AfterTOCHead{\singlespacing} % vom KOMA-Script-Autor empfohlen,
                               % hier aber ignoriert
\KOMAoptions{DIV=last}         % WICHTIG: Satzspiegel neu berechnen

%% Korrekter Satz von Zahlen und Einheiten
\usepackage{siunitx}

% Komma als Dezimaltrennzeichen im Text (unabhängig von der Eingabe)
\sisetup{output-decimal-marker={,}}

%% Logische Textauszeichnung
% fremdsprachige Begriffe kursiv
\newcommand*{\foreign}[1]{\emph{#1}}

% LaTeX-Paketnamen in Schreibmaschinenschrift (dicktengleich)
\newcommand*{\package}[1]{\texttt{#1}}

% (LaTeX-)Befehle in Schreibmaschinenschrift (dicktengleich)
\newcommand*{\command}[1]{\texttt{#1}}

% Datei- und Verzeichnisnamen in Schreibmaschinenschrift
\newcommand*{\filename}[1]{\texttt{#1}}