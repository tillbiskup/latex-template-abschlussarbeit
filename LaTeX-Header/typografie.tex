% LaTeX-Vorlage für Abschlussarbeiten (BSc/MSc/PhD)
%
% Diese Vorlage ist lediglich ein Vorschlag, der versucht, sowohl
% typografischen Ansprüchen ansatzweise gerecht zu werden als auch
% nutzbar zu sein.
%
% Jegliche Nutzung auf eigene Verantwortung.
%
% Copyright (c) 2018-23, Till Biskup

%% Mikrotypografie
\usepackage{microtype}

%% lmodern-Schrift
% alternativ: cm-super-Paket (nicht immer existent)
\ifPDFTeX
  \usepackage{lmodern}
\fi

%% Anführungszeichen sprachabhängig und "intelligent"
\usepackage[autostyle]{csquotes}

%% Abbildungsüber- und Tabellenunterschriften
\addtokomafont{caption}{\small}         % kleinere Schrift
\addtokomafont{captionlabel}{\bfseries} % fette Schrift
\setcapindent{0ex}                      % Keine hängenden Über-/Unterschriften

%% typografisch korrekte Tabellen
\usepackage{booktabs}
\KOMAoptions{captions=tableheading} % Tabellen haben Überschriften

%% Zeilenabstand erhöhen
\usepackage{setspace}
\setstretch{1.1}               % geringfügig erhöhter Zeilenabstand
%\AfterTOCHead{\singlespacing} % vom KOMA-Script-Autor empfohlen,
                               % hier aber ignoriert
\KOMAoptions{DIV=last}         % WICHTIG: Satzspiegel neu berechnen

%% Korrekter Satz von Zahlen und Einheiten
\usepackage{siunitx}
\sisetup{
  locale=DE,
  strict,
  list-final-separator = { und },
  list-pair-separator = { und },
  range-phrase = { bis }
}


%% Logische Textauszeichnung
% fremdsprachige Begriffe kursiv
\newcommand*{\foreign}[1]{\emph{#1}}

% LaTeX-Paketnamen in Schreibmaschinenschrift (dicktengleich)
\newcommand*{\package}[1]{\texttt{#1}}

% (LaTeX-)Befehle in Schreibmaschinenschrift (dicktengleich)
\newcommand*{\command}[1]{\texttt{#1}}

% Datei- und Verzeichnisnamen in Schreibmaschinenschrift
\newcommand*{\filename}[1]{\texttt{#1}}
