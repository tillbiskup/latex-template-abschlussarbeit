% Inoffizielle LaTeX-Vorlage für Bachelor-/Masterarbeiten
% an der Fakultät für Chemie und Pharmazie
% der Albert-Ludwigs-Universität Freiburg
%
% Diese Vorlage ist lediglich ein Vorschlag, der versucht, sowohl
% typografischen Ansprüchen ansatzweise gerecht zu werden als auch
% nutzbar zu sein.
%
% Jegliche Nutzung auf eigene Verantwortung.
%
% Copyright (c) 2018, Till Biskup

%% Einbinden von Quellcode
\usepackage{listings}

% Erweiterte Farbdefinitionen
\usepackage{xcolor}

% Unterstützte Sprachen sollten einmal geladen werden
\lstloadlanguages{[LaTeX]TeX,Python,Matlab}

\lstset{
  basicstyle=\footnotesize\ttfamily, % Standardschrift
  showstringspaces=false,            % Leerzeichen in Strings anzeigen?
  tabsize=2,                         % Größe von Tabs
  breaklines=true,                   % Zeilen werden umgebrochen
  prebreak=\dots,                    % drei Punkte vor dem Zeilenumbruch
  keywordstyle=\color{blue},         % Stil der Schlüsselworte
  frame=tb,                          % Rahmenlinie oben und unten
  belowcaptionskip=2ex               % Platz unterhalb der Überschrift
}

% Umlaute dürfen direkt in den Listings eingegeben werden
\lstset{
  literate=%
  {Ö}{{\"O}}1
  {Ä}{{\"A}}1
  {Ü}{{\"U}}1
  {ß}{{\ss}}1
  {ü}{{\"u}}1
  {ä}{{\"a}}1
  {ö}{{\"o}}1
  {~}{{\textasciitilde}}1,
  extendedchars=true
}

\lstdefinestyle{latex}{
  language=[LaTeX]TeX,
  commentstyle=\color{gray},       % Stil der Kommentare
  numbers=left,                    % Ort der Zeilennummern
  numberstyle=\tiny,               % Stil der Zeilennummern
  stepnumber=2,                    % Abstand zwischen den Zeilennummern
  numbersep=5pt,                   % Abstand der Nummern zum Text
  xleftmargin=17pt,
  framexleftmargin=17pt
}

% Stil für den Anhang mit kleinerer Schriftgröße
\lstdefinestyle{latex-appendix}{
  language=[LaTeX]TeX,
  basicstyle=\scriptsize\ttfamily, % Standardschrift
  commentstyle=\color{gray},       % Stil der Kommentare
  numbers=left,                    % Ort der Zeilennummern
  numberstyle=\tiny,               % Stil der Zeilennummern
  stepnumber=2,                    % Abstand zwischen den Zeilennummern
  numbersep=5pt,                   % Abstand der Nummern zum Text
  xleftmargin=17pt,
  framexleftmargin=17pt,
  morekeywords={lstset,KOMAoptions,addkomafont,setcapindent,%
    graphicspath,setstretch,DeclareGraphicsExtensions,addtokomafont,%
    DeclareMathAlphabet,addbibresource,DeclareFieldFormat,%
    renewbibmacro,lstloadlanguages,lstdefinestyle,hypersetup,color,%
    lvert,rvert,sisetup,appto}
}

% Das folgende ist aus 
% /usr/local/texlive/2018/texmf-dist/tex/latex/listings/listings-python.prf
% rauskopiert und angepasst
\definecolor{purple2}{RGB}{153,0,153} % there's actually no standard purple
\definecolor{green2}{RGB}{0,153,0}    % a darker green

\lstdefinestyle{python-idle-code}{%
  language=Python,                    % the language
  basicstyle=\footnotesize\ttfamily,    % size of the fonts for the code
  % Color settings to match IDLE style
  keywordstyle=\color{orange},        % core keywords
  keywordstyle={[2]\color{purple2}},  % built-ins
  stringstyle=\color{green2},
  commentstyle=\color{red},
  upquote=true,                       % requires textcomp
  numbers=left,                       % Ort der Zeilennummern
  numberstyle=\scriptsize,            % Stil der Zeilennummern
  stepnumber=2,                       % Abstand zwischen den Zeilennummern
  numbersep=5pt,                      % Abstand der Nummern zum Text
  xleftmargin=17pt,
  framexleftmargin=17pt
}
