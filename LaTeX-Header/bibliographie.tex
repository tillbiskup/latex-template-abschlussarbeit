% Inoffizielle LaTeX-Vorlage für Bachelor-/Masterarbeiten
% an der Fakultät für Chemie und Pharmazie
% der Albert-Ludwigs-Universität Freiburg
%
% Diese Vorlage ist lediglich ein Vorschlag, der versucht, sowohl
% typografischen Ansprüchen ansatzweise gerecht zu werden als auch
% nutzbar zu sein.
%
% Jegliche Nutzung auf eigene Verantwortung.
%
% Copyright (c) 2018, Till Biskup

%% Bibliographie
% Hinweis: \usepackage[autostyle]{csquotes} sollte vorher geladen werden
\usepackage[
	backend=biber,      % biber statt bibtex
	sorting=none,       % Reihenfolge in Zitierreihenfolge
	style=numeric-comp, % numerisch und im Text zusammengefasst
	isbn=false,         % keine ISBNs ausgeben
	url=false,          % keine URLs ausgeben
	doi=false,          % keine DOIs ausgeben
	eprint=false,       % keine Informationen zu ePrints ausgeben
	maxnames=10         % maximale Zahl von Autoren (danach "et al.")
]{biblatex}

% Einbinden der Bibliographiedatenbanken
% Die Reihenfolge ist mitunter entscheidend.
\addbibresource{Bibliographie/books.bib}
\addbibresource{Bibliographie/fulljrnl.bib}
\addbibresource{Bibliographie/articles.bib}

% Umdefinition mancher Ausgaben im Literaturverzeichnis
\DeclareFieldFormat[article]{title}{#1\isdot}
\renewbibmacro{in:}{%
  \ifentrytype{article}{}{\printtext{\bibstring{in}\intitlepunct}}}

% Flattersatz mit Silbentrennung
\usepackage[newcommands]{ragged2e}

% Literaturverzeichnis im linksbündigen Flattersatz
\appto{\bibsetup}{\sloppy\raggedright}