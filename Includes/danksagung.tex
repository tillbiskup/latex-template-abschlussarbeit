% Inoffizielle LaTeX-Vorlage für Bachelor-/Masterarbeiten
% an der Fakultät für Chemie und Pharmazie
% der Albert-Ludwigs-Universität Freiburg
%
% Diese Vorlage ist lediglich ein Vorschlag, der versucht, sowohl typografischen
% Ansprüchen ansatzweise gerecht zu werden als auch nutzbar zu sein.
%
% Jegliche Nutzung auf eigene Verantwortung.
%
% Copyright (c) 2018, Till Biskup

\chapter*{Danksagung}

Die Danksagung ist der Ort, an dem die Autorin oder der Autor ganz sie selbst sein können. Normalerweise beginnt eine Danksagung mit dem Arbeitskreisleiter und dem direkten Betreuer.

Mitunter lassen sich zwischen den Zeilen einer Danksagung sehr viele interessante Aspekte herauslesen \ldots

Wo eine Danksagung erscheint, ist eine Frage des persönlichen Geschmacks. Typischerweise erscheint sie ganz am Ende, nach den Anhängen und der Literatur. Selten findet man die Danksagung auch am Anfang einer Arbeit, das ist aber eher unüblich.

In jedem Fall sollte die Danksagung ein nicht nummeriertes Kapitel sein und nicht im Inhaltsverzeichnis erscheinen.

