% Inoffizielle LaTeX-Vorlage für Bachelor-/Masterarbeiten
% an der Fakultät für Chemie und Pharmazie
% der Albert-Ludwigs-Universität Freiburg
%
% Diese Vorlage ist lediglich ein Vorschlag, der versucht, sowohl
% typografischen Ansprüchen ansatzweise gerecht zu werden als auch
% nutzbar zu sein.
%
% Jegliche Nutzung auf eigene Verantwortung.
%
% Copyright (c) 2018, Till Biskup

%% Dokumentklasse und Optionen
\documentclass{scrreprt}       % Dokumentklasse aus dem KOMA-Script-Paket

% Einbinden der entsprechenden Header-Dateien
% aus dem Unterverzeichnis "LaTeX-Header"
% Die Reihenfolge ist nicht in jedem Fall egal.
% LaTeX-Vorlage für Abschlussarbeiten (BSc/MSc/PhD)
%
% Diese Vorlage ist lediglich ein Vorschlag, der versucht, sowohl
% typografischen Ansprüchen ansatzweise gerecht zu werden als auch
% nutzbar zu sein.
%
% Jegliche Nutzung auf eigene Verantwortung.
%
% Copyright (c) 2018-23, Till Biskup

\KOMAoptions{fontsize=12pt}    % Schriftgröße: 12 pt (Standard: 11 pt)
\KOMAoptions{BCOR=9.5mm}       % Bindekorrektur: 9.5 mm
\KOMAoptions{twoside}          % doppelseitiger Druck
\KOMAoptions{headsepline=true} % Trennlinie unter dem Kolumnentitel
\KOMAoptions{parskip=half}     % Abstand statt Einrückung für Absätze
\KOMAoptions{open=right}       % Kapitel auf rechter Seite beginnen

\pagestyle{headings}           % Seitenstil

% LaTeX-Vorlage für Abschlussarbeiten (BSc/MSc/PhD)
%
% Diese Vorlage ist lediglich ein Vorschlag, der versucht, sowohl
% typografischen Ansprüchen ansatzweise gerecht zu werden als auch
% nutzbar zu sein.
%
% Jegliche Nutzung auf eigene Verantwortung.
%
% Copyright (c) 2018-23, Till Biskup

\usepackage{iftex}

%% Zeichen- und Schriftkodierung
\ifPDFTeX
  \usepackage[utf8]{inputenc}    % Zeichenkodierung für das Dokument
  \usepackage[T1]{fontenc}       % Schriftkodierung
\else
  \usepackage{fontspec}
\fi

%% Sprachunterstützung
\ifPDFTeX
  \usepackage[ngerman]{babel}    % deutsche Sprache, neue Rechtschreibung
\else
  \usepackage{polyglossia}
  \setmainlanguage{german}
  \setotherlanguage{english}
\fi

% Inoffizielle LaTeX-Vorlage für Bachelor-/Masterarbeiten
% an der Fakultät für Chemie und Pharmazie
% der Albert-Ludwigs-Universität Freiburg
%
% Diese Vorlage ist lediglich ein Vorschlag, der versucht, sowohl typografischen
% Ansprüchen ansatzweise gerecht zu werden als auch nutzbar zu sein.
%
% Jegliche Nutzung auf eigene Verantwortung.
%
% Copyright (c) 2018, Till Biskup

%% Mikrotypografie
\usepackage{microtype}

%% Anführungszeichen sprachabhängig und "intelligent"
\usepackage[autostyle]{csquotes}

%% Abbildungsüber- und Tabellenunterschriften
\addtokomafont{caption}{\small}       % kleinere Schrift
\addtokomafont{captionlabel}{\textbf} % fette Schrift
\setcapindent{0ex}                    % Keine hängenden Über-/Unterschriften

%% typografisch korrekte Tabellen
\usepackage{booktabs}
\KOMAoptions{captions=tableheading} % Tabellen haben Überschriften

%% Zeilenabstand erhöhen
\usepackage{setspace}
\setstretch{1.1}               % geringfügig erhöhter Zeilenabstand
%\AfterTOCHead{\singlespacing} % vom KOMA-Script-Autor empfohlen, hier ignoriert
\KOMAoptions{DIV=last}         % WICHTIG: Satzspiegel neu berechnen

% LaTeX-Vorlage für Abschlussarbeiten (BSc/MSc/PhD)
%
% Diese Vorlage ist lediglich ein Vorschlag, der versucht, sowohl
% typografischen Ansprüchen ansatzweise gerecht zu werden als auch
% nutzbar zu sein.
%
% Jegliche Nutzung auf eigene Verantwortung.
%
% Copyright (c) 2018-23, Till Biskup

%% Grafiken einbinden
\usepackage{graphicx}
\usepackage{pgf}

% (relative) Pfade zu den Dateien
\graphicspath{{./Abbildungen/eigene/}{./Abbildungen/fremde/}}

% Reihenfolge der Dateiformate
\DeclareGraphicsExtensions{.pdf,.jpg,.png}

% Erweiterter Platz für Gleitobjekte auf der Seite
\renewcommand{\topfraction}{.8}
\renewcommand{\bottomfraction}{.5}

% Inoffizielle LaTeX-Vorlage für Bachelor-/Masterarbeiten
% an der Fakultät für Chemie und Pharmazie
% der Albert-Ludwigs-Universität Freiburg
%
% Diese Vorlage ist lediglich ein Vorschlag, der versucht, sowohl typografischen
% Ansprüchen ansatzweise gerecht zu werden als auch nutzbar zu sein.
%
% Jegliche Nutzung auf eigene Verantwortung.
%
% Copyright (c) 2018, Till Biskup

\usepackage{amsmath}

% Tensoren: serifenlos, fett, kursiv
\DeclareMathAlphabet{\mathbfsf}{\encodingdefault}{\sfdefault}{bx}{sl}
\newcommand{\tens}[1]{\mathbfsf{#1}}

% Dirac-Notation
\newcommand*{\bra}[1]{\left\langle#1\right\rvert}
\newcommand*{\ket}[1]{\left\lvert#1\right\rangle}
\newcommand*{\braket}[2]{\langle#1\lvert#2\rangle}

% Inoffizielle LaTeX-Vorlage für Bachelor-/Masterarbeiten
% an der Fakultät für Chemie und Pharmazie
% der Albert-Ludwigs-Universität Freiburg
%
% Diese Vorlage ist lediglich ein Vorschlag, der versucht, sowohl
% typografischen Ansprüchen ansatzweise gerecht zu werden als auch
% nutzbar zu sein.
%
% Jegliche Nutzung auf eigene Verantwortung.
%
% Copyright (c) 2018, Till Biskup

%% Bibliographie
% Hinweis: \usepackage[autostyle]{csquotes} sollte vorher geladen werden
\usepackage[
	backend=biber,      % biber statt bibtex
	sorting=none,       % Reihenfolge in Zitierreihenfolge
	style=numeric-comp, % numerisch und im Text zusammengefasst
	isbn=false,         % keine ISBNs ausgeben
	url=false,          % keine URLs ausgeben
	doi=false,          % keine DOIs ausgeben
	eprint=false,       % keine Informationen zu ePrints ausgeben
	maxnames=10         % maximale Zahl von Autoren (danach "et al.")
]{biblatex}

% Einbinden der Bibliographiedatenbanken
% Die Reihenfolge ist mitunter entscheidend.
\addbibresource{Bibliographie/books.bib}
\addbibresource{Bibliographie/fulljrnl.bib}
\addbibresource{Bibliographie/articles.bib}

% Umdefinition mancher Ausgaben im Literaturverzeichnis
\DeclareFieldFormat[article]{title}{#1\isdot}
\renewbibmacro{in:}{%
  \ifentrytype{article}{}{\printtext{\bibstring{in}\intitlepunct}}}

% Flattersatz mit Silbentrennung
\usepackage[newcommands]{ragged2e}

% Literaturverzeichnis im linksbündigen Flattersatz
\appto{\bibsetup}{\sloppy\raggedright}
% LaTeX-Vorlage für Abschlussarbeiten (BSc/MSc/PhD)
%
% Diese Vorlage ist lediglich ein Vorschlag, der versucht, sowohl
% typografischen Ansprüchen ansatzweise gerecht zu werden als auch
% nutzbar zu sein.
%
% Jegliche Nutzung auf eigene Verantwortung.
%
% Copyright (c) 2018-23, Till Biskup

%% Einbinden von Quellcode
\usepackage{listings}

% Erweiterte Farbdefinitionen
\usepackage{xcolor}

% Unterstützte Sprachen sollten einmal geladen werden
\lstloadlanguages{[LaTeX]TeX,Python,Matlab}

\lstset{
  basicstyle=\footnotesize\ttfamily, % Standardschrift
  showstringspaces=false,            % Leerzeichen in Strings anzeigen?
  tabsize=2,                         % Größe von Tabs
  breaklines=true,                   % Zeilen werden umgebrochen
  prebreak=\dots,                    % drei Punkte vor dem Zeilenumbruch
  keywordstyle=\color{blue},         % Stil der Schlüsselworte
  frame=tb,                          % Rahmenlinie oben und unten
  belowcaptionskip=2ex               % Platz unterhalb der Überschrift
}

% Umlaute dürfen direkt in den Listings eingegeben werden
\lstset{
  literate=%
  {Ö}{{\"O}}1
  {Ä}{{\"A}}1
  {Ü}{{\"U}}1
  {ß}{{\ss}}1
  {ü}{{\"u}}1
  {ä}{{\"a}}1
  {ö}{{\"o}}1
  {~}{{\textasciitilde}}1,
  extendedchars=true
}

\lstdefinestyle{latex}{
  language=[LaTeX]TeX,
  commentstyle=\color{gray},       % Stil der Kommentare
  numbers=left,                    % Ort der Zeilennummern
  numberstyle=\tiny,               % Stil der Zeilennummern
  stepnumber=2,                    % Abstand zwischen den Zeilennummern
  numbersep=5pt,                   % Abstand der Nummern zum Text
  xleftmargin=17pt,
  framexleftmargin=17pt
}

% Stil für den Anhang mit kleinerer Schriftgröße
\lstdefinestyle{latex-appendix}{
  language=[LaTeX]TeX,
  basicstyle=\scriptsize\ttfamily, % Standardschrift
  commentstyle=\color{gray},       % Stil der Kommentare
  numbers=left,                    % Ort der Zeilennummern
  numberstyle=\tiny,               % Stil der Zeilennummern
  stepnumber=2,                    % Abstand zwischen den Zeilennummern
  numbersep=5pt,                   % Abstand der Nummern zum Text
  xleftmargin=17pt,
  framexleftmargin=17pt,
  morekeywords={lstset,KOMAoptions,addkomafont,setcapindent,%
    graphicspath,setstretch,DeclareGraphicsExtensions,addtokomafont,%
    DeclareMathAlphabet,addbibresource,DeclareFieldFormat,%
    renewbibmacro,lstloadlanguages,lstdefinestyle,hypersetup,color,%
    lvert,rvert,sisetup,appto}
}

% Das folgende ist aus 
% /usr/local/texlive/2018/texmf-dist/tex/latex/listings/listings-python.prf
% rauskopiert und angepasst
\definecolor{purple2}{RGB}{153,0,153} % there's actually no standard purple
\definecolor{green2}{RGB}{0,153,0}    % a darker green

\lstdefinestyle{python-idle-code}{%
  language=Python,                    % the language
  basicstyle=\footnotesize\ttfamily,    % size of the fonts for the code
  % Color settings to match IDLE style
  keywordstyle=\color{orange},        % core keywords
  keywordstyle={[2]\color{purple2}},  % built-ins
  stringstyle=\color{green2},
  commentstyle=\color{red},
  upquote=true,                       % requires textcomp
  numbers=left,                       % Ort der Zeilennummern
  numberstyle=\scriptsize,            % Stil der Zeilennummern
  stepnumber=2,                       % Abstand zwischen den Zeilennummern
  numbersep=5pt,                      % Abstand der Nummern zum Text
  xleftmargin=17pt,
  framexleftmargin=17pt
}


% LaTeX-Vorlage für Abschlussarbeiten (BSc/MSc/PhD)
%
% Diese Vorlage ist lediglich ein Vorschlag, der versucht, sowohl
% typografischen Ansprüchen ansatzweise gerecht zu werden als auch
% nutzbar zu sein.
%
% Jegliche Nutzung auf eigene Verantwortung.
%
% Copyright (c) 2018-23, Till Biskup

%% Abkürzungen
\usepackage[printonlyused]{acronym}


% In dieser Datei werden die Metadaten für die Titelseite definiert.
% Inoffizielle LaTeX-Vorlage für Bachelor-/Masterarbeiten
% an der Fakultät für Chemie und Pharmazie
% der Albert-Ludwigs-Universität Freiburg
%
% Diese Vorlage ist lediglich ein Vorschlag, der versucht, sowohl
% typografischen Ansprüchen ansatzweise gerecht zu werden als auch
% nutzbar zu sein.
%
% Jegliche Nutzung auf eigene Verantwortung.
%
% Copyright (c) 2018, Till Biskup

%% Definition von Befehlen für die Titelseite
\newcommand*{\Titel}{\LaTeX{} für Naturwissenschaftler:\\ Ansprechender Text- und Formelsatz\\ von Abschlussarbeiten}
\newcommand*{\Typ}{Bachelorarbeit}
\newcommand*{\Grad}{Bachelor of Science}
\newcommand*{\Fach}{Chemie}
\newcommand*{\Fakultaet}{Fakultät für Chemie und Pharmazie}
\newcommand*{\Name}{Hans Wurst}
\newcommand*{\Geburtsort}{Hintertupfingen}
\newcommand*{\Jahr}{2018}


% Das Paket "hyperref" sollte GANZ AM ENDE der Präambel geladen werden.
% Insbesondere sollten die Definitionen für die Titelseite vorher geladen
% werden, da Teile dieser Informationen in die Metadaten des PDF-Dokuments
% übernommen werden.
\input{LaTeX-Header/hyperref}

\begin{document}

% Die einzelnen Kapitel werden als separate Dateien eingebunden.
% All diese Dateien liegen im Verzeichnis "Includes".

\pagenumbering{roman}

% LaTeX-Vorlage für Abschlussarbeiten (BSc/MSc/PhD)
%
% Diese Vorlage ist lediglich ein Vorschlag, der versucht, sowohl
% typografischen Ansprüchen ansatzweise gerecht zu werden als auch
% nutzbar zu sein.
%
% Jegliche Nutzung auf eigene Verantwortung.
%
% Copyright (c) 2018-23, Till Biskup

\begin{titlepage}
\setcounter{page}{-1}
\centering
\begin{Large}
\bfseries\Titel\par
\end{Large}

\vspace*{\fill}

\begin{large}
\textbf{\Typ}\\
zur Erlangung des Grades eines\\
\Grad\ \Fach\par
\end{large}

\vspace*{\fill}

\begin{large}
vorgelegt von\\
\textbf{\Name}\\
aus \Geburtsort\par
\end{large}

\vspace*{\fill}

\includegraphics{unilogo}

\vspace*{\fill}

\begin{large}
\Fakultaet\\
\Universitaet\\
\Ort\\
\Jahr\par
\end{large}
\end{titlepage}


%\chapter*{Publikationen}

Teile dieser Arbeit wurden veröffentlicht:

\begin{list}{}{\setlength\leftmargin{0ex}}
\item A. Erstautor, B. Koautor und C. Chef. Ein paar dumme Gedanken zu \LaTeX{} in den Naturwissenschaften. \emph{Journal für halbherzige Forschung} 3 (2022), S. 3--28.

\item A. Erstautor, B. Koautor und C. Chef. Ein paar weitere dumme Gedanken zu \LaTeX{} in den Naturwissenschaften. \emph{Journal für halbherzige Forschung} 4 (2023), S. 313--321.
\end{list}


\tableofcontents

% Kann weggelassen werden
\listoffigures

% Kann weggelassen werden
\listoftables

% Kann weggelassen werden
\lstlistoflistings

% LaTeX-Vorlage für Abschlussarbeiten (BSc/MSc/PhD)
%
% Diese Vorlage ist lediglich ein Vorschlag, der versucht, sowohl
% typografischen Ansprüchen ansatzweise gerecht zu werden als auch
% nutzbar zu sein.
%
% Jegliche Nutzung auf eigene Verantwortung.
%
% Copyright (c) 2018-23, Till Biskup

\chapter*{Abkürzungsverzeichnis}

\begin{acronym}

\acro{DFT}{Dichtefunktionaltheorie}

\acro{ESR}{Elektronenspinresonanz}

\acro{HOMO}{Highest Occupied Molecular Orbital}

\acro{LUMO}{Lowest Unoccupied Molecular Orbital}

\acro{NMR}{Nuclear Magnetic Resonance}

\end{acronym}


\cleardoublepage
\pagenumbering{arabic}

% LaTeX-Vorlage für Abschlussarbeiten (BSc/MSc/PhD)
%
% Diese Vorlage ist lediglich ein Vorschlag, der versucht, sowohl
% typografischen Ansprüchen ansatzweise gerecht zu werden als auch
% nutzbar zu sein.
%
% Jegliche Nutzung auf eigene Verantwortung.
%
% Copyright (c) 2018-23, Till Biskup

\chapter{Einleitung}
\label{ch:einleitung}

Das von Leslie Lamport in den 1980er Jahren auf der Basis von \TeX{} \cite{knut-bams-1-337} entwickelte Textsatzsystem \LaTeX{} \cite{kopka-h-2000} erfreut sich nicht zuletzt wegen seiner herausragenden Fähigkeiten zum mathematischen Formelsatz einer großen Beliebtheit insbesondere in den physikalisch orientierten Wissenschaften. Für viele Studierende in der Physikalischen Chemie ist die Bachelorarbeit der erste Anlass, sich mit \LaTeX{} auseinanderzusetzen, und auch wenn es gute Dokumentationen und online verfügbare Hilfsmittel gibt, tauchen immer wieder ähnliche Fragen auf.


\section{Ziel dieser Arbeit}

Aus eigener langjähriger Erfahrung, u.a. als Betreuer diverser Abschlussarbeiten, entstand die Idee, im Rahmen eines Einführungskurses in den Textsatz mit \LaTeX{} eine Vorlage für Abschlussarbeiten zu entwickeln, die sich zumindest für Bachelor- und Masterarbeiten eignet und den Studierenden weitestgehend die Aufgabe abnimmt, sich um das Layout zu kümmern, und ihnen stattdessen erlaubt, sich auf die Inhalte ihrer Abschlussarbeit zu fokussieren.

Das Ergebnis dieser Bemühungen ist das vorliegende Dokument, das als Versuchsobjekt dient, um mittels der KOMA-Script-Klasse \package{scrreprt} \cite{kohm-m-2014} ein Layout zu verwirklichen, das sich für Bachelor- und Masterarbeiten zunächst einmal in der Fakultät für Chemie und Pharmazie der Albert-Ludwigs-Universität Freiburg oder noch konkreter im Institut für Physikalische Chemie eben jener Fakultät eignet.

Der Autor dieser Vorlage ist kein Typograf, dafür aber ein langjähriger passionierter Nutzer von \LaTeX{}, der mit dieser Vorlage seine ersten systematischen Gehversuche bei der Verwendung der KOMA-Script-Klassen macht. Das Dokument entstand parallel zum Kurs \enquote{\href{https://www.till-biskup.de/de/lehre/latex/index}{\LaTeX{} für Naturwissenschaftler: Ansprechender Text- und Formelsatz von Abschlussarbeiten}}, den der Autor erstmalig im Sommersemester 2018 gab. Die Folien zum Kurs sowie weiterführende Informationen finden sich auf der Webseite zum Kurs: \url{https://www.till-biskup.de/de/lehre/latex/index}.

Nun stellt sich natürlich schon die Frage, warum man denn das Layout einer Abschlussarbeit nicht den betreffenden Kandidatinnen und Kandidaten selbst überlässt. Die Antwort ist mehrgeteilt. Zum Einen ist diese Vorlage, ebenso wie der Kurs, in dessen Kontext sie entstanden ist, lediglich ein Angebot und ein Ausgangspunkt für die eigene Beschäftigung mit \LaTeX{} und seinen Möglichkeiten. Zum Anderen tauchen aus der Erfahrung immer wieder die gleichen Probleme auf, und auch wenn diese Vorlage fern davon ist, die eine gültige Lösung zu präsentieren, so müht sie sich doch, \emph{eine} gangbare Lösung aufzuzeigen, die aus typografischen Gesichtspunkten zumindest vertretbar ist. Dabei werden durchaus Kompromisse eingegangen, insbesondere was Zeilenbreite (eher breit) und Zeilenabstand (leicht vergrößert) angeht, die mit Sicherheit nicht dem Ideal der Typografie für Bücher entsprechen.


\section{Hinweise zum Umgang}

Die \LaTeX{}-Quellen, die diesem Dokument zugrunde liegen, sind auf mehrere Dateien und Verzeichnisse aufgeteilt und geben in Grundzügen die Gliederung einer typischen Abschlussarbeit in der Physikalischen Chemie wieder. Entsprechend können die Quellen direkt als Vorlage für die eigene Abschlussarbeit dienen, die Inhalte (wie der Text dieser Einleitung) werden dann natürlich durch eigene Inhalte ersetzt.

Gleichzeitig erscheint es dem Autor aber sinnvoll, eine originale Kopie der Quellen dieses Dokuments zur Verfügung zu haben, da in den einzelnen Kapiteln durchaus ernst zu nehmende Beispiele für bestimmte, immer wiederkehrende Aspekte des Textsatzes auftauchen, sei es die Einbindung von Abbildungen und Tabellen oder der mathematische Formelsatz.

Letztlich ist \LaTeX{} ziemlich kulant, was die Formatierung des Quelltextes eines Dokuments angeht. Trotzdem sollte auch der Quelltext so übersichtlich wie möglich formatiert werden, zumal er häufiger gelesen als geschrieben wird.

Zumindest momentan ersetzt dieses Dokument in keiner Weise die Inhalte und Folien des Kurses, in dessen Rahmen es entstand. Trotz diverser Hinweise meist zu Beginn der jeweiligen Kapitel sollte man es entsprechend nicht als eine Verschriftlichung der Kursunterlagen verstehen. Weiterhin sind die Beispiele für bestimmte Aspekte des Textsatzes momentan auch noch recht dürftig, hier ist in Zukunft ggf. mit einer deutlichen Erweiterung zu rechnen.


\section{Verfügbarkeit}

Sollte das Projekt erfolgreich sein und alle wesentlichen Aspekte, die in einer typischen Bachelor- und Masterarbeit auftreten, abdecken, wird es am Ende in der einen oder anderen Art zur Verfügung gestellt. Wer es dann als \enquote{Rundum-Sorglos-Paket} verwenden möchte, ohne dabei nachzudenken, ist für sein eigenes Unglück verantwortlich. Eingedenk der eigenen Erfahrung, dass derlei Vorlagen ein gewisses Eigenleben entwickeln, sobald sie in irgendeiner Form zur Verfügung gestellt werden, ist in die Erstellung dieses Dokuments eine gewisse Sorgfalt geflossen. Trotzdem übernimmt sein Autor keinerlei Garantie für irgendwelche Ergebnisse oder Folgen.



\include{Includes/theorie}

% Inoffizielle LaTeX-Vorlage für Bachelor-/Masterarbeiten
% an der Fakultät für Chemie und Pharmazie
% der Albert-Ludwigs-Universität Freiburg
%
% Diese Vorlage ist lediglich ein Vorschlag, der versucht, sowohl typografischen
% Ansprüchen ansatzweise gerecht zu werden als auch nutzbar zu sein.
%
% Jegliche Nutzung auf eigene Verantwortung.
%
% Copyright (c) 2018, Till Biskup

\chapter{Materialien und Methoden}



% Inoffizielle LaTeX-Vorlage für Bachelor-/Masterarbeiten
% an der Fakultät für Chemie und Pharmazie
% der Albert-Ludwigs-Universität Freiburg
%
% Diese Vorlage ist lediglich ein Vorschlag, der versucht, sowohl
% typografischen Ansprüchen ansatzweise gerecht zu werden als auch
% nutzbar zu sein.
%
% Jegliche Nutzung auf eigene Verantwortung.
%
% Copyright (c) 2018, Till Biskup

\chapter{Ergebnisse und Diskussion}
\label{ch:ergebnisse_diskussion}

Ergebnisse und Diskussion können in zwei separate, aufeinander folgende Kapitel aufgeteilt werden. Die eigene Erfahrung spricht allerdings eher dagegen.

Werden Ergebnisse und Diskussion, wie hier vorgegeben, in einem Kapitel gemeinsam abgehandelt, ist es umso wichtiger darauf zu achten, dass zunächst die eigentlichen Ergebnisse beschrieben und dann erst im Kontext anderer Arbeiten diskutiert werden. Eine reine Darstellung der Ergebnisse in Abbildungen und Tabellen ist \emph{nicht} ausreichend.

Auch die -- für gewöhnlich in Tabellen übersichtlich aufbereiteten -- Parameter von Simulationen, die an gemessene Daten angepasst wurden, gehören \foreign{per se} erst einmal zu den Daten, die textlich beschrieben werden sollten.



\section{Umgang mit Abbildungen}

Abbildungen sollten nach Möglichkeit immer als vektorisierte Grafikformate vorliegen. Die Erfahrung zeigt, dass es möglich ist, Abbildungen von Anfang an so anzulegen, dass sie sowohl in der Abschlussarbeit als auch in einem zugehörigen Vortrag verwendet werden können. Die zur Beschriftung von Abbildungen verwendete Schriftgröße ist für beide Fälle passend.

Wer diesen Ansatz verfolgt, kann ggf. das Abbildungsverzeichnis auf eine Ebene mit der Abschlussarbeit und dem Vortrag verschieben und die relativen Pfade in der Präambel-Datei zur Handhabung von Abbildungen (\filename{grafiken.tex}, vgl. Listing~\ref{lst:grafiken}, S.~\pageref{lst:grafiken}) entsprechend anpassen.

Das Verzeichnis für die Abbildungen ist weiter unterteilt in eigene und fremde Abbildungen. Für fremde Abbildungen ist es eine gute Idee, in einer Datei mit gleichem Grundnamen (und evtl. der Endung \filename{.txt}) Hinweise zum jeweiligen Urheberrecht abzulegen, die sich dann ggf. in die Abbildungsunterschrift übernehmen lassen.

Dieses Dokument ist von seinem Layout so angelegt, dass als Breite des Satzspiegels\footnote{Als Satzspiegel oder Schriftspiegel wird die nutzbare Fläche einer Buchseite bzw. allgemein einer Textseite eines gedruckten Dokuments bezeichnet.} recht genau 15~cm entspricht. Werden die Abbildungen entsprechend im Format 15:10 oder 15:9 angelegt, lassen sie sich unskaliert direkt in dieses Dokument einbinden und auf Seitenbreite skaliert in eine Vortragsfolie (bei Verwendung des Layouts der Albert-Ludwigs-Universität Freiburg und des \package{beamer}-Pakets für \LaTeX{}).

Wie schon für Tabellen in Kapitel~\ref{ch:material_methoden} angesprochen, haben Abbildungen, auf die aus dem Text nicht verwiesen wird, keine Daseinsberechtigung. Entsprechend sei hier auf Abb.~\ref{fig:beispiel} verwiesen.

\begin{figure}[t]
\includegraphics{Beispielabbildung}
\caption[Abbildungen haben immer eine Bildunterschrift.]{\textbf{Abbildungen haben immer eine Bildunterschrift.} Außerdem ist es hilfreich, wenn die Bildunterschrift die Abbildung soweit erklärt, dass der geneigte Leser nicht erst noch große Teile des Textes lesen muss, um sie zu verstehen. Kurz: Abbildungen sollten gemeinsam mit ihrer Unterschrift für sich stehen.}
\label{fig:beispiel}
\end{figure}


% Inoffizielle LaTeX-Vorlage für Bachelor-/Masterarbeiten
% an der Fakultät für Chemie und Pharmazie
% der Albert-Ludwigs-Universität Freiburg
%
% Diese Vorlage ist lediglich ein Vorschlag, der versucht, sowohl
% typografischen Ansprüchen ansatzweise gerecht zu werden als auch
% nutzbar zu sein.
%
% Jegliche Nutzung auf eigene Verantwortung.
%
% Copyright (c) 2018, Till Biskup

\chapter{Zusammenfassung und Ausblick}
\label{ch:zusammenfassung_ausblick}

Ziel dieser Arbeit war die Erstellung einer \LaTeX{}-Vorlage für Abschlussarbeiten, zunächst ganz spezifisch am Institut für Physikalische Chemie der Albert-Ludwigs-Universität Freiburg, die den Nutzern erlaubt, sich weitestgehend auf die Inhalte der jeweiligen Arbeit zu konzentrieren und sich über ein Mindestmaß hinaus nicht mit Details der Formatierung beschäftigen zu müssen.


\section{Zentrale Aspekte des verfolgten Ansatzes}

\paragraph{Vordefinierte Titelseite}

Die Formatierung der Titelseite ist vordefiniert und entspricht den Vorgaben von Seiten der Fakultät und des Prüfungsamtes. Lediglich die Metadaten, also Titel, Autor, akademischer Grad etc. müssen über entsprechende Befehle in der Präambel an den jeweiligen spezifischen Kontext angepasst werden. Das offizielle Logo der Albert-Ludwigs-Universität Freiburg ist ebenfalls Teil der Vorlage und wird entsprechend auf der Titelseite eingebunden.


\paragraph{Auslagerung der Kapitel in eigene Dateien}

Jedes Kapitel wird in eine eigene Datei ausgelagert und über den Befehl \command{include} eingebunden. All diese eingebundenen Dateien befinden sich in einem eigenen Verzeichnis, \filename{Includes}, weshalb nur das Hauptdokument und die durch die diversen Kompilierungen erzeugten Hilfs- und Ausgabedateien auf der obersten Ebene übrig bleiben. Das erhöht immens die Übersicht. Darüber hinaus unterstützen moderne \LaTeX{}-Editoren die Definition eines Hauptdokuments, so dass die Kompilierung bequem auch dann erfolgen kann, wenn man gerade eines der Kapitel bearbeitet. Alternativ steht dafür das \filename{Makefile} zur Verfügung, das durch den Aufruf von \command{make} auf der Kommandozeile\footnote{auch als \enquote{Eingabeaufforderung}, \enquote{cmd} oder \enquote{Terminal} bezeichnet} die Kompilierung erlaubt.


\paragraph{Aufteilung der Präambel}

Ähnlich wie die Kapitel in einzelne Dateien aufgeteilt wurden, besteht die Präambel ebenfalls aus einzelnen, thematisch unterteilten Dateien. Diese Modularisierung erhöht einerseits die Übersichtlichkeit und erlaubt es andererseits, nur die jeweils notwendigen Aspekte einzubinden. Die Reihenfolge der einzelnen Teile der Präambel sind nicht vollständig beliebig, Hinweise finden sich in der Präambel des Hauptdokuments und in den einzelnen Dateien.



\section{Ideen für künftige Erweiterungen}

\paragraph{Weitere Beispiele}

Bislang finden sich in diesem Dokument nur relativ wenige konkrete Beispiele, die \LaTeX{}-Quellcode für die Lösung immer wiederkehrender Fragestellungen zur Verfügung stellen. Ohne zu sehr inhaltlich Aspekte vorweg zu nehmen (eine Einführung in die Magnetresonanz als Beispiel für den mathematischen Formelsatz wäre eher wenig sinnvoll), ließe sich hier sicherlich noch manche Ergänzung vornehmen.


\paragraph{Detailliertere Hinweise zur Gliederung der Arbeit}

Vielleicht wäre es hilfreich, zu jedem der Kapitel einführend ein wenig zu beschreiben, worauf man sinnvollerweise achten sollte und wie die jeweiligen Kapitel aufgebaut werden können. Das entspräche in Teilen der Verschriftlichung entsprechender Foliensätze des zugrundeliegenden Kurses.


\paragraph{Erklärung der einzelnen Layout-Entscheidungen}

Viele der getroffenen Entscheidungen bzgl. des grunsätzlichen Layouts sind das Ergebnis intensiven Nachdenkens seitens seines Autors. Zu einem kleinen Teil finden sich bereits Hinweise dazu im Anhang der Arbeit bei der Auflistung der einzelnen Teile der Präambel. Trotzdem wäre es, vermutlich inbesondere bzgl. der Einstellungen zu Satzspiegel und anderen Elementen, hilfreich, detaillierter zu beschreiben, wie es dazu kam.


\paragraph{Dokumentation der von der Vorlage bereitgestellten Strukturen}

Nicht nur die Titelseite ist eine Struktur, die in dieser Form explizit nur von dieser Vorlage bereitgestellt wird. Das betrifft ebenso eine Reihe von Befehlen zur logischen Textauszeichnung im mathematischen Formelsatz und im regulären Textsatz. Eine konkrete Dokumentation dieser Befehle und Strukturen erleichterte mit Sicherheit die Nutzbarkeit der Vorlage.



\appendix

% Inoffizielle LaTeX-Vorlage für Bachelor-/Masterarbeiten
% an der Fakultät für Chemie und Pharmazie
% der Albert-Ludwigs-Universität Freiburg
%
% Diese Vorlage ist lediglich ein Vorschlag, der versucht, sowohl
% typografischen Ansprüchen ansatzweise gerecht zu werden als auch
% nutzbar zu sein.
%
% Jegliche Nutzung auf eigene Verantwortung.
%
% Copyright (c) 2018, Till Biskup

\chapter{Verwendete Präambel-Dateien}
\label{ch:anhang-header}

Eine Idee, die bei der Erstellung dieses Dokumentes umgesetzt wurde, ist die Aufteilung nicht nur der einzelnen Kapitel des Inhaltes auf eigene Dateien in einem eigenen Verzeichnis, sondern in gleicher Weise die Aufteilung der Dokumentpräambel. Entsprechend befindet sich im Verzeichnis \filename{LaTeX-Header} eine Reihe von Dateien, die thematisch unterschiedliche Aspekte der Präambel abdecken.

Diese Dateien seien nachfolgend hier dokumentiert. Neben den Kommentaren im Quellcode selbst dienen die jeweiligen Überschriften ebenfalls der weiteren Dokumentation und Beschreibung der jeweils realisierten Ideen.

Der Quellcode dieses Kapitels dient gleichzeitig dazu zu zeigen, wie sich Quellcode von Auswertungsprogrammen einer Abschlussarbeit, die z.B. in der Physikalischen Chemie angefertigt wurde, in einem Anhang in die Arbeit integrieren ließe. Bei dieser Gelegenheit sei noch einmal daran erinnert, dass selbst im Rahmen einer Abschlussarbeit geschriebene Programme, die zur Auswertung der Daten verwendet wurden, aus Gründen der Nachvollziehbarkeit und Reproduzierbarkeit auf jeden Fall archiviert werden müssen. Darüber hinaus ist es ein viel zu selten umgesetzter guter Standard, sie als Anhang in die Abschlussarbeit aufzunehmen.

Noch ein letzter Hinweis: Der Schriftgrad der Quellcode-Listings ist bewusst relativ klein gehalten, um die Zeilen nicht unnötig umbrechen zu müssen. Für Quellcode im Haupttext der Arbeit sollte eine größere Schrift gewählt werden.


\lstinputlisting[style=latex-appendix,firstline=13,caption={[\LaTeX{}-Präambel für das allgemeine Layout.]\textbf{\LaTeX{}-Präambel für das allgemeine Layout des Dokuments.} Diese Präambel wird in der Regel als erste im Dokument eingelesen. Die hier vorgenommenen Einstellungen sind das Ergebnis ausführlicher Tests.},label={lst:layout}]{LaTeX-Header/layout}


\lstinputlisting[style=latex-appendix,firstline=13,caption={[\LaTeX{}-Präambel für die Spracheinstellungen.]\textbf{\LaTeX{}-Präambel für die Spracheinstellungen des Dokuments.} Neben der Definition der Zeichen- und Schriftkodierung für das \LaTeX-Dokument wird die eigentliche Dokumentsprache definiert. Kommen mehrere Sprachen zum Einsatz, sollten sie der Reihe nach dem Paket \package{babel} übergeben werden. In diesem Fall ist die letzte angegebene Sprache die Hauptsprache des Dokuments.},label={lst:sprache}]{LaTeX-Header/sprache}


\lstinputlisting[style=latex-appendix,firstline=13,caption={[\LaTeX{}-Präambel für typografische Aspekte.]\textbf{\LaTeX{}-Präambel für typografische Aspekte.} Zugegeben sind viele der Einstellungen, die in diesem Teil der Präambel vorgenommen werden, subjektiv und nicht alle unter dem Gesichtspunkt der professionellen Typografie perfekt (insbesondere der etwas erhöhte Zeilenabstand und die Hervorhebung von Absätzen durch Abstand statt Einrückung). Sie haben sich aber für naturwissenschaftliche Abschlussarbeiten bewährt.},label={lst:typografie}]{LaTeX-Header/typografie}


\lstinputlisting[style=latex-appendix,firstline=13,caption={[\LaTeX{}-Präambel für das Einbinden von Abbildungen.]\textbf{\LaTeX{}-Präambel für das Einbinden von Abbildungen.} Neben dem Einbinden des Pakets \package{graphicx} ist hier insbesondere die Definition der relativen Pfade zu den Abbildungen von Interesse.},label={lst:grafiken}]{LaTeX-Header/grafiken}


\lstinputlisting[style=latex-appendix,firstline=13,caption={[\LaTeX{}-Präambel für den mathematischen Formelsatz.]\textbf{\LaTeX{}-Präambel für den mathematischen Formelsatz.} Das Paket \package{amsmath} ist für den Formelsatz eigentlich unerlässlich und sollte immer geladen werden, sobald mathematische Formeln gesetzt werden müssen. Die weiteren Aspekte betreffen die zusätzliche Definition von mathematischen Strukturen mit möglichst sprechenden Befehlen, die der logischen Textauszeichnung dienen und die Lesbarkeit des \LaTeX{}-Quelltextes entsprechend erhöhen.},label={lst:mathematik}]{LaTeX-Header/mathematik}


\lstinputlisting[style=latex-appendix,firstline=13,caption={[\LaTeX{}-Präambel für die Bibliographie.]\textbf{\LaTeX{}-Präambel für die Bibliographie.} Hier werden sowohl allgemeine Einstellungen des Paketes \package{biblatex} vorgenommen als auch die entsprechenden Literaturdatenbanken definiert. Der linksbündige Flattersatz ist für die Ausgabe des Literaturverzeichnisses sehr empfohlen. Für die korrekte Silbentrennugn sorgt das Paket \package{ragged2e}.},label={lst:bibliographie}]{LaTeX-Header/bibliographie}


\lstinputlisting[style=latex-appendix,firstline=13,caption={[\LaTeX{}-Präambel für die Einbindung von Quellcodebeispielen.]\textbf{\LaTeX{}-Präambel für die Einbindung von Quellcodebeispielen.} Aufgrund der zahlreichen Einstellungsmöglichkeiten ist die Präambel zur Handhabung von Quellcodebeispielen die ausführlichste in diesem Dokument. Nur ein Teil der Möglichkeiten wird hier genutzt, für weitere Informationen sei auf die Dokumentation zum Paket \package{listings} verwiesen.},label={lst:listings}]{LaTeX-Header/listings}


\lstinputlisting[style=latex-appendix,firstline=13,caption={[\LaTeX{}-Präambel für die Handhabung von Abkürzungen\\ und das Abkürzungsverzeichnis.]\textbf{\LaTeX{}-Präambel für die Handhabung von Abkürzungen und das Abkürzungsverzeichnis.} Diese Präambel ist äußerst kurz, aus Gründen der Modularität und weil nicht in jeder Arbeit ein Abkürzungsverzeichnis verwendet werden wird, ist sie aber in eine eigene Datei ausgelagert worden.},label={lst:abkuerzungen}]{LaTeX-Header/abkuerzungen}


\lstinputlisting[style=latex-appendix,firstline=13,caption={[\LaTeX{}-Präambel für die Definition der Daten auf der Titelseite.]\textbf{\LaTeX{}-Präambel für die Definition der Daten auf der Titelseite.} Die eigentliche Titelseite wird ihrerseits einfach nur als erste Datei in das Dokument eingebunden, aber nicht verändert. Alle inhaltlichen Einstellungen finden in der Präambel statt. Die hier in der Präambel eingegebenen Metadaten wie Autor und Titel finden dann auch Verwendung beim Laden des Paketes \package{hyperref}.},label={lst:titelseite}]{LaTeX-Header/titelseite}


\lstinputlisting[style=latex-appendix,firstline=13,caption={[\LaTeX{}-Präambel für das Paket \package{hyperref}.]\LaTeX{}-Präambel für das Paket \package{hyperref}. Das Paket \package{hyperref} sollte \emph{ganz am Ende} der Präambel geladen werden. Insbesondere sollten die Definitionen für die Titelseite vorher geladen werden, da Teile dieser Informationen in die Metadaten des PDF-Dokuments übernommen werden.},label={lst:hyperref}]{LaTeX-Header/hyperref}



\printbibliography[heading=bibintoc]

% Inoffizielle LaTeX-Vorlage für Bachelor-/Masterarbeiten
% an der Fakultät für Chemie und Pharmazie
% der Albert-Ludwigs-Universität Freiburg
%
% Diese Vorlage ist lediglich ein Vorschlag, der versucht, sowohl typografischen
% Ansprüchen ansatzweise gerecht zu werden als auch nutzbar zu sein.
%
% Jegliche Nutzung auf eigene Verantwortung.
%
% Copyright (c) 2018, Till Biskup

\chapter*{Danksagung}

Die Danksagung ist der Ort, an dem die Autorin oder der Autor ganz sie selbst sein können. Normalerweise beginnt eine Danksagung mit dem Arbeitskreisleiter und dem direkten Betreuer.

Mitunter lassen sich zwischen den Zeilen einer Danksagung sehr viele interessante Aspekte herauslesen \ldots

Wo eine Danksagung erscheint, ist eine Frage des persönlichen Geschmacks. Typischerweise erscheint sie ganz am Ende, nach den Anhängen und der Literatur. Selten findet man die Danksagung auch am Anfang einer Arbeit, das ist aber eher unüblich.

In jedem Fall sollte die Danksagung ein nicht nummeriertes Kapitel sein und nicht im Inhaltsverzeichnis erscheinen.



\end{document}